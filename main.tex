\documentclass[11pt, letterpaper, titlepage]{article}

\usepackage{geometry}
\geometry{left=1.5cm,right=1.5cm,top=1.5cm,bottom=2.5cm}

\usepackage{setspace}
\onehalfspacing

\usepackage{fancyhdr}

\usepackage{amsmath}

\usepackage{amssymb}

\usepackage{booktabs}

\usepackage{pifont}

\pagestyle{fancy}
\lhead{}
\rhead{}
\lfoot{ECEN 629 Applied Convex Optimization}
\cfoot{\thepage}
\rfoot{Assignment 1 @Lei Wang}
\renewcommand{\headrulewidth}{0pt}
\renewcommand{\headwidth}{\textwidth}
\renewcommand{\footrulewidth}{0.4pt}
\newcommand{\RomanNumeralCaps}[1]
    {\MakeUppercase{\romannumeral #1}}
    
\begin{document}

\begin{enumerate}
    \item %Q1
    
    \begin{equation*}
        \begin{aligned}
            a_{1}t + ... + a_{k}t^{k} &\geq 1 \\
            -a_{1}t - ... - a_{k}t^{k} &\leq -1 \text{, also} \\
            a_{1}\text{cos} t + ... + a_{k} \text{cos} kt &\leq 1 \text{, hence} \\
            \sum_{l=1}^{k} a_{l}(\text{cos} lt - t^{l}) &\leq 0
        \end{aligned}
    \end{equation*}
    
    Take two points, $ x $ and $ y $ out of the set, apply the definition:
    
    \begin{equation*}
        A = \theta \sum_{l=1}^{k} x_{l}(\text{cos} lt - t^{l}) + (1-\theta) \sum_{l=1}^{k} y_{l}(\text{cos} lt - t^{l})
    \end{equation*}
    
    Both summation are $ \leq 0 $, also $ 0 \leq \theta \leq 1 $ and $ 0 \leq 1 - \theta \leq 1 $. Therefore $ A \leq 0 $ and the set is convex.
    
    \item %Q2
    
    \begin{enumerate}
        \item $ (c, d) = (-ka, ka), k\in \mathbb{R}^{+}$. The line segment connecting $ c $ and $ d $ is bisected by the plane $ a x = b $. $ c $ is on the side of the plane that contains the halfspace. $ d $ is not on the side of the plane that contains the halfspace.
        
        \item If the planes $ a x = 1 $ and $ a' x = 1 $ are parallel to each other, plane $ a x = 1 $ bisects the line segment connecting $ c $ and $ d $ and plane $ a' x = 1 $ bisects the line segment connecting $ c $ and $ e $. The set is the intersection between the two halfspaces, i.e. the space between the two parallel planes. $ a $ and $ a' $ should be pointing at exactly the opposite directions and may have different lengths. Let:
        
        \begin{equation*}
            \begin{aligned}
                a = - k a'
            \end{aligned}
        \end{equation*}
        
        Therefore:
        
        \begin{equation*}
            \begin{aligned}
                c & = - k a = a' \\
                d &= k a \\
                e & = a' \\
                k a + a' &= 0 \\
                - k a - a' &= 0 \text{, for }k \in \mathbb{R}^{+}
            \end{aligned}
        \end{equation*}
        
         If the planes $ a x = 1 $ and $ a' x = 1 $ are perpendicular to each other, plane $ a x = 1 $ bisects the line segment connecting $ c $ and $ d $ and plane $ a' x = 1 $ bisects the line segment connecting $ c $ and $ e $. The set is the intersection between the two halfspaces, i.e. the space between the two perpendicular planes. $ a $ and $ a' $ should be pointing at perpendicular directions and may have different lengths. Let:
    
        \begin{equation*}
            \begin{aligned}
                ||a||_2 &= k ||a'||_2 \text{, which } k \in \mathbb{R}^{+} \text{, and } \\
                a \cdot a' &= 0
            \end{aligned}
        \end{equation*}
        
        Therefore:
        
        \begin{equation*}
            \begin{aligned}
                c &= -a = - k a' \\
                d &= a \\
                e &= k a' \text{, which } k \in \mathbb{R}^{+}
            \end{aligned}
        \end{equation*}
    \end{enumerate}
    
    The perpendicular case solution also applies to the case when two planes intersect but not perpendicular.
    
    \item %Q3
    
    \begin{enumerate}
        \item 
        
        \item
        
        \item
    \end{enumerate}
    
    \item %Q4
    
    \begin{enumerate}
        \item Let $ S_3 $ be the sum of $ S_1 $ and $ S_2 $:
        
        \begin{equation*}
            \begin{aligned}
                a \text{, } b \in S_1 \text{, } c \text{, } d \in S_2 \text{, } e \text{, } f \in S_3 \text{, and } e &= a + c \text{, } f = b + d \\
                \theta e + (1 - \theta) f = \theta (a + c) + (1 - \theta) (b + d) &= ( \theta a + (1 - \theta) b) + ( \theta c + (1 - \theta) d) \in S_1 + S_2 = S_3
            \end{aligned}
        \end{equation*}
        
        The set is convex.
        
        \item
    \end{enumerate}
    
    \item %Q5
    
    \begin{enumerate}
        \item The set $ S $ is the plane separating $ C $ and $ D $. Based on the separating hyperplane theorem, $ C $ and $ D $ are two disjoint and convex sets, $ S $ is the plane separating the two sets. Hence $ S $ is convex.
        
        \item The set $ S $ is not convex. As required by the separating hyperplane theorem, two sets need to be convex and disjoint.
        
        \item The set $ S $ is not convex and may be empty. If there were overlaps between set $ C $ and $ D $, separation between the two sets is not possible.
    \end{enumerate}
    
    \newpage
    
    \item %Q6
    
    $ f $ is concave, according to the first order condition:
    
    \begin{equation*}
        \begin{aligned}
            f(x) &\leq f(a) + f'(a)(x-a) \\
            \frac{f(x) - f(a)}{x-a} &\leq f'(a) \\
            f(x) &\leq f(b) + f'(b)(x-b) \\
            \frac{f(x) - f(b)}{x-b} &\leq f'(b) \\
            a &< b \text{, therefore,} \\
            f'(a) &> f'(b) \\
            \frac{f(x) - f(a)}{x-a} &\geq \frac{f(x) - f(b)}{x-b} \\
            \frac{f(x) - f(a)}{x-a} &\geq \frac{f(b) - f(x)}{b-x} \\
            f(b) &\leq f(a) + f'(a)(b-a) \\
            \frac{f(b) - f(a)}{b-a} &\leq f'(a) \\
            f(a) &\geq f(b) + f'(b)(a-b) \\
            \frac{f(a) - f(b)}{a-b} &\geq f'(b) \text{, put together,} \\
            \frac{f(x) - f(a)}{x-a} &\geq \frac{f(a) - f(b)}{a-b} \geq \frac{f(b) - f(x)}{b-x} \\
            \frac{f(x) - f(a)}{x-a} &\geq \frac{f(b) - f(a)}{b-a} \geq \frac{f(b) - f(x)}{b-x} \\
        \end{aligned}
    \end{equation*}
    
    If the function is twice differentiable, then a global maximum exists and the function is smooth and continuous.
    
    \item
    
    \item
    
    \item
    
    \item
\end{enumerate}

\end{document}
